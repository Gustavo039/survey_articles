%Version 2.1 April 2023
% See section 11 of the User Manual for version history
%
%%%%%%%%%%%%%%%%%%%%%%%%%%%%%%%%%%%%%%%%%%%%%%%%%%%%%%%%%%%%%%%%%%%%%%
%%                                                                 %%
%% Please do not use \input{...} to include other tex files.       %%
%% Submit your LaTeX manuscript as one .tex document.              %%
%%                                                                 %%
%% All additional figures and files should be attached             %%
%% separately and not embedded in the \TeX\ document itself.       %%
%%                                                                 %%
%%%%%%%%%%%%%%%%%%%%%%%%%%%%%%%%%%%%%%%%%%%%%%%%%%%%%%%%%%%%%%%%%%%%%

\documentclass[sn-basic,pdflatex]{sn-jnl}

%%%% Standard Packages
%%<additional latex packages if required can be included here>

\usepackage{graphicx}%
\usepackage{multirow}%
\usepackage{amsmath,amssymb,amsfonts}%
\usepackage{amsthm}%
\usepackage{mathrsfs}%
\usepackage[title]{appendix}%
\usepackage{xcolor}%
\usepackage{textcomp}%
\usepackage{manyfoot}%
\usepackage{booktabs}%
\usepackage{algorithm}%
\usepackage{algorithmicx}%
\usepackage{algpseudocode}%
\usepackage{listings}%
%%%%

%%%%%=============================================================================%%%%
%%%%  Remarks: This template is provided to aid authors with the preparation
%%%%  of original research articles intended for submission to journals published
%%%%  by Springer Nature. The guidance has been prepared in partnership with
%%%%  production teams to conform to Springer Nature technical requirements.
%%%%  Editorial and presentation requirements differ among journal portfolios and
%%%%  research disciplines. You may find sections in this template are irrelevant
%%%%  to your work and are empowered to omit any such section if allowed by the
%%%%  journal you intend to submit to. The submission guidelines and policies
%%%%  of the journal take precedence. A detailed User Manual is available in the
%%%%  template package for technical guidance.
%%%%%=============================================================================%%%%

%% Per the spinger doc, new theorem styles can be included using built in style, 
%% but it seems the don't work so commented below
%\theoremstyle{thmstyleone}%
\newtheorem{theorem}{Theorem}%  meant for continuous numbers
%%\newtheorem{theorem}{Theorem}[section]% meant for sectionwise numbers
%% optional argument [theorem] produces theorem numbering sequence instead of independent numbers for Proposition
\newtheorem{proposition}[theorem]{Proposition}%
%%\newtheorem{proposition}{Proposition}% to get separate numbers for theorem and proposition etc.

%% \theoremstyle{thmstyletwo}%
\theoremstyle{remark}
\newtheorem{example}{Example}%
\newtheorem{remark}{Remark}%

%% \theoremstyle{thmstylethree}%
\theoremstyle{definition}
\newtheorem{definition}{Definition}%



\raggedbottom




% tightlist command for lists without linebreak
\providecommand{\tightlist}{%
  \setlength{\itemsep}{0pt}\setlength{\parskip}{0pt}}

% From pandoc table feature
\usepackage{longtable,booktabs,array}
\usepackage{calc} % for calculating minipage widths
% Correct order of tables after \paragraph or \subparagraph
\usepackage{etoolbox}
\makeatletter
\patchcmd\longtable{\par}{\if@noskipsec\mbox{}\fi\par}{}{}
\makeatother
% Allow footnotes in longtable head/foot
\IfFileExists{footnotehyper.sty}{\usepackage{footnotehyper}}{\usepackage{footnote}}
\makesavenoteenv{longtable}




\begin{document}


\title[Article Title runing]{Planos amostrais complexos na estimação da
média das notas de português}

%%=============================================================%%
%% Prefix	-> \pfx{Dr}
%% GivenName	-> \fnm{Joergen W.}
%% Particle	-> \spfx{van der} -> surname prefix
%% FamilyName	-> \sur{Ploeg}
%% Suffix	-> \sfx{IV}
%% NatureName	-> \tanm{Poet Laureate} -> Title after name
%% Degrees	-> \dgr{MSc, PhD}
%% \author*[1,2]{\pfx{Dr} \fnm{Joergen W.} \spfx{van der} \sur{Ploeg} \sfx{IV} \tanm{Poet Laureate}
%%                 \dgr{MSc, PhD}}\email{iauthor@gmail.com}
%%=============================================================%%

\author[1]{\fnm{Pedro} \spfx{Henrique Corrêa de} \sur{Almeida} }

\author[1]{\fnm{Gustavo} \spfx{Almeida} \sur{Silva} }



  \affil*[1]{\orgdiv{Estatística}, \orgname{UFJF}, \orgaddress{\city{Juiz
de Fora}, \country{Brasil}, \state{MG}}}

\abstract{O estudo de simulação realizado nesta pesquisa envolve a
geração de 1.000 replicações de designs de amostragem estratificada e de
múltiplos conglomerados. Cada replicação considera diferentes cenários,
como tamanhos variados de cluster e níveis de correlação dentro do
cluster. Os dados simulados são então analisados usando técnicas
estatísticas apropriadas para avaliar o viés, erros padrão e outras
medidas relevantes para cada projeto de amostragem. Os resultados desta
pesquisa contribuem para a compreensão dos pontos fortes e limitações
das pesquisas estratificadas e de múltiplos conglomerados. O estudo
destaca a importância de considerar desenhos de amostragem complexos e
suas métricas associadas para obter estimativas confiáveis e robustas. O
conhecimento obtido com este trabalho pode ajudar pesquisadores e
profissionais na seleção de estratégias de amostragem apropriadas para
seus contextos de pesquisa específicos.}

\keywords{Amostragem, Plano Amostral Complexo, Amostragem
estratificada, Amostragem por conglomerado, Simulação Monte Carlo}



\maketitle

\newpage

\hypertarget{resumo}{%
\section{Resumo}\label{resumo}}

Este trabalho apresenta um estudo de simulação sobre métodos de
amostragem complexa para a estimação da média amostral. Os métodos
analisados são baseados em amostragem conglomerada em 1, 2 e 3 estágios.
O objetivo é comparar o desempenho desses métodos em termos de
eficiência e precisão da estimativa.

A amostragem complexa é amplamente utilizada em pesquisas em que a
população de interesse possui uma estrutura hierárquica ou está dividida
em subpopulações distintas. A amostragem conglomerada é uma técnica
comumente aplicada nesse contexto, em que a população é dividida em
conglomerados e, em seguida, uma amostra é selecionada em cada
conglomerado.

Neste estudo, são simulados diferentes cenários com base em parâmetros
de amostragem realistas. São considerados os métodos de amostragem
conglomerada em 1, 2 e 3 estágios, nos quais a seleção dos conglomerados
e das unidades amostrais é realizada de forma sequencial.

Através das simulações, são comparados os estimadores das médias
amostrais obtidos pelos diferentes métodos, levando em consideração a
variância da estimativa e a eficiência em relação ao tamanho da amostra.
Além disso, são avaliados possíveis Viéses de estimadores e a precisão
das estimativas em cada estágio da amostragem conglomerada.

Os resultados das simulações fornecem insights valiosos sobre a
adequação e o desempenho dos métodos de amostragem conglomerada em
diferentes estágios. Espera-se que este estudo contribua para a
compreensão das complexidades da amostragem em pesquisas com estrutura
hierárquica e auxilie pesquisadores na escolha do método de amostragem
mais apropriado para suas necessidades.

\hypertarget{introduuxe7uxe3o}{%
\section{Introdução}\label{introduuxe7uxe3o}}

A amostragem desempenha um papel fundamental na estatística, permitindo
aos pesquisadores obterem informações sobre uma população a partir de
uma amostra representativa. Através de métodos estatísticos robustos, é
possível extrapolar conclusões precisas e confiáveis sobre a população
em geral. No entanto, a amostragem muitas vezes enfrenta desafios
práticos, como a seleção adequada das unidades amostrais e a
consideração de complexidades inerentes a certos planos de amostragem.

De forma geral, é amplamente reconhecido na teoria da amostragem que,
embora o esquema de amostragem aleatória simples (AAS) seja teoricamente
simples, na prática, é pouco utilizado devido às restrições
orçamentárias e à busca por métodos probabilísticos que forneçam
informações mais precisas. Além disso, é comum encontrar dificuldades na
obtenção de cadastros adequados para o AAS, bem como lidar com situações
de não resposta, o que requer considerar observações com pesos desiguais
\citep{skinner2005design}. A especificação inadequada na análise do
plano amostral selecionado também pode resultar em estimativas
enViésadas, destacando a importância de estudar metodologias que levem
em conta o esquema de amostragem adotado.

Este artigo tem como objetivo explorar a interseção entre a amostragem
em estatística e a simulação computacional, destacando como essa
abordagem combinada pode contribuir para aprimorar a qualidade das
inferências estatísticas. Serão apresentados conceitos fundamentais da
amostragem, incluindo diferentes métodos de seleção amostral e as
respectivas propriedades, e, em seguida, será discutido como a simulação
computacional pode ser aplicada para investigar essas técnicas em
contextos específicos.

Ao integrar a simulação computacional à amostragem estatística, os
pesquisadores podem explorar virtualmente uma ampla gama de cenários de
amostragem, considerando diferentes planos amostrais, tamanhos de
amostra e distribuições populacionais. Além disso, a simulação permite a
avaliação de métricas de desempenho, como viés e erro padrão, fornecendo
insights valiosos sobre a precisão e a eficiência dos métodos de
amostragem em diferentes contextos.

\hypertarget{metodologia}{%
\section{Metodologia}\label{metodologia}}

O objetivo deste trabalho é comparar diferentes planos de amostragem em
estágios complexos, como a amostragem estratificada e a amostragem
conglomerada. Para realizar essa comparação, foi conduzido um estudo de
simulação. O estudo tem como propósito investigar e avaliar o desempenho
desses diferentes planos amostrais em termos de eficiência, precisão e
viés. Através da simulação, é possível criar cenários controlados que
permitem analisar o impacto de cada plano amostral em diferentes
características da população. Com base nos resultados obtidos na
simulação, será possível identificar quais planos de amostragem são mais
adequados para determinados contextos e auxiliar na tomada de decisões
estatísticas mais embasadas.

Para isso, foi utilizado o conjunto de dados: \textbf{Alunos.txt}, que
se trata de dados sobre notas de alunos na prova de portugues. Os dados
são populacionais. Assim, diferentes métodos de amostragem complexa
foram avaliados utilizando esse conjunto de dados.

O conjunto de dados possui 6 variáveis, são elas:

\begin{longtable}[]{@{}
  >{\raggedright\arraybackslash}p{(\columnwidth - 2\tabcolsep) * \real{0.3158}}
  >{\raggedright\arraybackslash}p{(\columnwidth - 2\tabcolsep) * \real{0.6842}}@{}}
\toprule()
\begin{minipage}[b]{\linewidth}\raggedright
Variável
\end{minipage} & \begin{minipage}[b]{\linewidth}\raggedright
Descrição
\end{minipage} \\
\midrule()
\endhead
Aluno & ID do aluno \\
Rede & Rede de ensino \\
Escola & ID da escola \\
Turma & ID da turma \\
Port & Nota no teste de portugues de cada aluno, é tambem a variável de
interesse desse trabalho \\
\bottomrule()
\end{longtable}

Os metodos estudados foram:

\begin{itemize}
\item
  Amostragem Estratificada

  \begin{itemize}
  \tightlist
  \item
    Foram testados estratificação por Rede e estratificação por Escola
  \end{itemize}
\item
  Amostragem Conglomerada

  \begin{itemize}
  \item
    1 estágio por Escolas
  \item
    1 estágio por Turmas
  \item
    2 estágios: UPA-Escolas, USA-Turmas
  \item
    3 estágios: UPA-Escolas, USA-Turmas, UTA-Alunos
  \end{itemize}
\item
  Amostragem Conglomerada com PPT Poisson

  \begin{itemize}
  \tightlist
  \item
    1 estágio por Escolas, tamanho via número de turmas
  \item
    1 estágio por Escolas, tamanho via número de alunos
  \end{itemize}
\end{itemize}

\hypertarget{estudo-de-simulauxe7uxe3o}{%
\section{Estudo de Simulação}\label{estudo-de-simulauxe7uxe3o}}

Considerou-se como variavel de interesse a média da variavel Port com
transformação logaritmo natural, ou seja, a variavel estimada via
diferentes metodos de amostragem complexa foi: \(ln({Port})\)

Para a cada plano amostral, foram replicadas 1000 vezes amostras de
tamanho 500 e 750, para cada replicação foram calculadas as estimativas
pontuais, o erro padrão e o intervalo de confiança de \(95\%\)

Para avaliar o desempenho de cada plano amostral. foram consideradas
metricas como \textbf{Víes, Erro-padrão e Erro Quadrático Médio}.

O verdadeiro valor da variavel estimada é de:

\[
\frac{\sum_{i=1}^n{ln(Port_i)}}{n} = 6.218181
\]

O conhecimento de tal valor é importantíssimo para o calculo do viés e
consequente a decisão sobre o plano maostral mais adequado para o
problema.

\hypertarget{amostragem-estratificada}{%
\subsection{Amostragem Estratificada}\label{amostragem-estratificada}}

A amostragem estratificada é uma técnica valiosa que permite uma seleção
mais precisa e representativa da amostra, considerando as
heterogeneidades presentes na população. Ao estratificar a população em
subgrupos e selecionar uma amostra de cada estrato, é possível obter
estimativas mais confiáveis e insights mais detalhados sobre os
diferentes grupos presentes na população de interesse.

Nesse contexto, trabalhou-se com duas divisões de estratos: Rede e
Escola. Primeiramente foi realizada 1000 replicações com um tamanho
amostral igual a 500, após isso realizou-se novamente 1000 replicações
com 750. Vemos um ótimo dos 3 tipos de alocação, onde todos apresentam
um EQM extremamente baixo. O intervalo de confiança considerado foi de
\(95%
\), assim vemos que o metodo de alocação Proporcional foi aquele que
mais se aproximou desse valor. Os demais metodos apresentaram taxa de
rejeição inferior ao nivel de significancia definida.

\hypertarget{estratificada-por-rede}{%
\subsubsection{Estratificada por Rede}\label{estratificada-por-rede}}

Após aplicar as 1000 replicações utilizando o método de Monte Carlo.
Seguem as médias das estimativas produzidas utilizando cada uma das
alocações.

\hypertarget{n-500}{%
\paragraph{n = 500}\label{n-500}}

\begin{longtable}[]{@{}
  >{\raggedright\arraybackslash}p{(\columnwidth - 10\tabcolsep) * \real{0.2241}}
  >{\raggedright\arraybackslash}p{(\columnwidth - 10\tabcolsep) * \real{0.1552}}
  >{\raggedright\arraybackslash}p{(\columnwidth - 10\tabcolsep) * \real{0.1897}}
  >{\raggedright\arraybackslash}p{(\columnwidth - 10\tabcolsep) * \real{0.1207}}
  >{\raggedright\arraybackslash}p{(\columnwidth - 10\tabcolsep) * \real{0.1724}}
  >{\raggedright\arraybackslash}p{(\columnwidth - 10\tabcolsep) * \real{0.1379}}@{}}
\toprule()
\begin{minipage}[b]{\linewidth}\raggedright
Alocação
\end{minipage} & \begin{minipage}[b]{\linewidth}\raggedright
\(\hat{\theta}\)
\end{minipage} & \begin{minipage}[b]{\linewidth}\raggedright
\(\hat{EP(\theta)}\)
\end{minipage} & \begin{minipage}[b]{\linewidth}\raggedright
\(IC(95\%)\subset \theta\)
\end{minipage} & \begin{minipage}[b]{\linewidth}\raggedright
Viés
\end{minipage} & \begin{minipage}[b]{\linewidth}\raggedright
EQM
\end{minipage} \\
\midrule()
\endhead
Uniforme & 6.218338 & 0.0104576 & 0.966 & 0.0001568 & 0.0001094 \\
Proporcional & 6.218293 & 0.0090885 & 0.951 & 0.0001121 & 0.0000826 \\
Neyman & 6.218179 & 0.0090841 & 0.964 & -0.0000020 & 0.0000825 \\
\bottomrule()
\end{longtable}

\hypertarget{n-750}{%
\paragraph{n = 750}\label{n-750}}

\begin{longtable}[]{@{}
  >{\raggedright\arraybackslash}p{(\columnwidth - 10\tabcolsep) * \real{0.2241}}
  >{\raggedright\arraybackslash}p{(\columnwidth - 10\tabcolsep) * \real{0.1552}}
  >{\raggedright\arraybackslash}p{(\columnwidth - 10\tabcolsep) * \real{0.1897}}
  >{\raggedright\arraybackslash}p{(\columnwidth - 10\tabcolsep) * \real{0.1207}}
  >{\raggedright\arraybackslash}p{(\columnwidth - 10\tabcolsep) * \real{0.1724}}
  >{\raggedright\arraybackslash}p{(\columnwidth - 10\tabcolsep) * \real{0.1379}}@{}}
\toprule()
\begin{minipage}[b]{\linewidth}\raggedright
Alocação
\end{minipage} & \begin{minipage}[b]{\linewidth}\raggedright
Estimativas
\end{minipage} & \begin{minipage}[b]{\linewidth}\raggedright
ErroPadrão
\end{minipage} & \begin{minipage}[b]{\linewidth}\raggedright
\(IC(95\%)\subset \theta\)
\end{minipage} & \begin{minipage}[b]{\linewidth}\raggedright
Viés
\end{minipage} & \begin{minipage}[b]{\linewidth}\raggedright
EQM
\end{minipage} \\
\midrule()
\endhead
Uniforme & 6.217622 & 0.0085606 & 0.968 & -0.0005597 & 7.36e-05 \\
Proporcional & 6.218224 & 0.0074278 & 0.960 & 0.0000427 & 5.52e-05 \\
Neyman & 6.217922 & 0.0074324 & 0.966 & -0.0002592 & 5.53e-05 \\
\bottomrule()
\end{longtable}

Com base no resultados obtidos a partir do estudo de simulação, podemos
verificar que os intervalos de confiança de 95\% conteram o valor real,
aproximadamente, 95\% das vezes, como era o esperado. Além disso, para
todos os métodos, tivemos um erro quadrático médio baixo,

\hypertarget{estratificada-por-escola}{%
\subsubsection{Estratificada por
Escola}\label{estratificada-por-escola}}

\begin{longtable}[]{@{}
  >{\raggedright\arraybackslash}p{(\columnwidth - 10\tabcolsep) * \real{0.2241}}
  >{\raggedright\arraybackslash}p{(\columnwidth - 10\tabcolsep) * \real{0.1552}}
  >{\raggedright\arraybackslash}p{(\columnwidth - 10\tabcolsep) * \real{0.1897}}
  >{\raggedright\arraybackslash}p{(\columnwidth - 10\tabcolsep) * \real{0.1207}}
  >{\raggedright\arraybackslash}p{(\columnwidth - 10\tabcolsep) * \real{0.1724}}
  >{\raggedright\arraybackslash}p{(\columnwidth - 10\tabcolsep) * \real{0.1379}}@{}}
\toprule()
\begin{minipage}[b]{\linewidth}\raggedright
Alocação
\end{minipage} & \begin{minipage}[b]{\linewidth}\raggedright
Estimativas
\end{minipage} & \begin{minipage}[b]{\linewidth}\raggedright
ErroPadrão
\end{minipage} & \begin{minipage}[b]{\linewidth}\raggedright
\(IC(95\%)\subset \theta\)
\end{minipage} & \begin{minipage}[b]{\linewidth}\raggedright
Viés
\end{minipage} & \begin{minipage}[b]{\linewidth}\raggedright
EQM
\end{minipage} \\
\midrule()
\endhead
Uniforme & 6.218014 & 0.0094822 & 0.981 & -0.0001673 & 8.99e-05 \\
Proporcional & 6.217734 & 0.0084041 & 0.983 & -0.0004475 & 7.08e-05 \\
Neyman & 6.217978 & 0.0086743 & 0.986 & -0.0002035 & 7.53e-05 \\
\bottomrule()
\end{longtable}

\begin{longtable}[]{@{}
  >{\raggedright\arraybackslash}p{(\columnwidth - 10\tabcolsep) * \real{0.2241}}
  >{\raggedright\arraybackslash}p{(\columnwidth - 10\tabcolsep) * \real{0.1552}}
  >{\raggedright\arraybackslash}p{(\columnwidth - 10\tabcolsep) * \real{0.1897}}
  >{\raggedright\arraybackslash}p{(\columnwidth - 10\tabcolsep) * \real{0.1207}}
  >{\raggedright\arraybackslash}p{(\columnwidth - 10\tabcolsep) * \real{0.1724}}
  >{\raggedright\arraybackslash}p{(\columnwidth - 10\tabcolsep) * \real{0.1379}}@{}}
\toprule()
\begin{minipage}[b]{\linewidth}\raggedright
Alocação
\end{minipage} & \begin{minipage}[b]{\linewidth}\raggedright
Estimativas
\end{minipage} & \begin{minipage}[b]{\linewidth}\raggedright
ErroPadrão
\end{minipage} & \begin{minipage}[b]{\linewidth}\raggedright
\(IC(95\%)\subset \theta\)
\end{minipage} & \begin{minipage}[b]{\linewidth}\raggedright
Viés
\end{minipage} & \begin{minipage}[b]{\linewidth}\raggedright
EQM
\end{minipage} \\
\midrule()
\endhead
Uniforme & 6.218394 & 0.0081765 & 0.982 & 0.0002125 & 6.69e-05 \\
Proporcional & 6.218155 & 0.0070644 & 0.991 & -0.0000268 & 4.99e-05 \\
Neyman & 6.218010 & 0.0072053 & 0.984 & -0.0001719 & 5.19e-05 \\
\bottomrule()
\end{longtable}

Os 3 metodos apresentaram um erro quadrático médio baixo, porém o
intervalo de confiança não obteve desempenho desejado. Definido um nivel
de confiança de \(5%
\), viu-se que os três métodos apresentaram um nível de rejeição menor
que o esperado. Buscando uma melhor convergência, ambos os metodos foram
replicados 1000 vezes novamente, porém fixando um tamanho amostral
maior, igual a 750

\hypertarget{amostragem-conglomerada}{%
\subsection{Amostragem Conglomerada}\label{amostragem-conglomerada}}

\hypertarget{conclusuxe3o}{%
\section{Conclusão}\label{conclusuxe3o}}

\renewcommand\refname{Referências}
\bibliography{bibliography.bib}


\end{document}
