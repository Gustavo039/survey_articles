%Version 2.1 April 2023
% See section 11 of the User Manual for version history
%
%%%%%%%%%%%%%%%%%%%%%%%%%%%%%%%%%%%%%%%%%%%%%%%%%%%%%%%%%%%%%%%%%%%%%%
%%                                                                 %%
%% Please do not use \input{...} to include other tex files.       %%
%% Submit your LaTeX manuscript as one .tex document.              %%
%%                                                                 %%
%% All additional figures and files should be attached             %%
%% separately and not embedded in the \TeX\ document itself.       %%
%%                                                                 %%
%%%%%%%%%%%%%%%%%%%%%%%%%%%%%%%%%%%%%%%%%%%%%%%%%%%%%%%%%%%%%%%%%%%%%

\documentclass[sn-basic,pdflatex]{sn-jnl}

%%%% Standard Packages
%%<additional latex packages if required can be included here>

\usepackage{graphicx}%
\usepackage{multirow}%
\usepackage{amsmath,amssymb,amsfonts}%
\usepackage{amsthm}%
\usepackage{mathrsfs}%
\usepackage[title]{appendix}%
\usepackage{xcolor}%
\usepackage{textcomp}%
\usepackage{manyfoot}%
\usepackage{booktabs}%
\usepackage{algorithm}%
\usepackage{algorithmicx}%
\usepackage{algpseudocode}%
\usepackage{listings}%
%%%%

%%%%%=============================================================================%%%%
%%%%  Remarks: This template is provided to aid authors with the preparation
%%%%  of original research articles intended for submission to journals published
%%%%  by Springer Nature. The guidance has been prepared in partnership with
%%%%  production teams to conform to Springer Nature technical requirements.
%%%%  Editorial and presentation requirements differ among journal portfolios and
%%%%  research disciplines. You may find sections in this template are irrelevant
%%%%  to your work and are empowered to omit any such section if allowed by the
%%%%  journal you intend to submit to. The submission guidelines and policies
%%%%  of the journal take precedence. A detailed User Manual is available in the
%%%%  template package for technical guidance.
%%%%%=============================================================================%%%%

%% Per the spinger doc, new theorem styles can be included using built in style, 
%% but it seems the don't work so commented below
%\theoremstyle{thmstyleone}%
\newtheorem{theorem}{Theorem}%  meant for continuous numbers
%%\newtheorem{theorem}{Theorem}[section]% meant for sectionwise numbers
%% optional argument [theorem] produces theorem numbering sequence instead of independent numbers for Proposition
\newtheorem{proposition}[theorem]{Proposition}%
%%\newtheorem{proposition}{Proposition}% to get separate numbers for theorem and proposition etc.

%% \theoremstyle{thmstyletwo}%
\theoremstyle{remark}
\newtheorem{example}{Example}%
\newtheorem{remark}{Remark}%

%% \theoremstyle{thmstylethree}%
\theoremstyle{definition}
\newtheorem{definition}{Definition}%



\raggedbottom




% tightlist command for lists without linebreak
\providecommand{\tightlist}{%
  \setlength{\itemsep}{0pt}\setlength{\parskip}{0pt}}





\begin{document}


\title[Article Title runing]{Article Title}

%%=============================================================%%
%% Prefix	-> \pfx{Dr}
%% GivenName	-> \fnm{Joergen W.}
%% Particle	-> \spfx{van der} -> surname prefix
%% FamilyName	-> \sur{Ploeg}
%% Suffix	-> \sfx{IV}
%% NatureName	-> \tanm{Poet Laureate} -> Title after name
%% Degrees	-> \dgr{MSc, PhD}
%% \author*[1,2]{\pfx{Dr} \fnm{Joergen W.} \spfx{van der} \sur{Ploeg} \sfx{IV} \tanm{Poet Laureate}
%%                 \dgr{MSc, PhD}}\email{iauthor@gmail.com}
%%=============================================================%%

\author*[1,2]{\pfx{Dr.} \fnm{Leading} \spfx{van} \sur{Author} \sfx{III} \tanm{Poet
Laureate} \dgr{MSc,
PhD}}\email{\href{mailto:abc@def}{\nolinkurl{abc@def}}}

\author[2]{\fnm{Second} \sur{Author} }



  \affil*[1]{\orgdiv{Department}, \orgname{Organization}, \orgaddress{\city{City}, \country{Country}, \postcode{100190}, \state{State}, \street{Street}}}
  \affil*[2]{\orgname{Other Organisation}}

\abstract{\textbf{Purpose}: The abstract serves both as a general
introduction to the topic and as a brief, non-technical summary of the
main results and their implications. The abstract must not include
subheadings (unless expressly permitted in the journal's Instructions to
Authors), equations or citations. As a guide the abstract should not
exceed 200 words. Most journals do not set a hard limit however authors
are advised to check the author instructions for the journal they are
submitting to.

\textbf{Methods:} The abstract serves both as a general introduction to
the topic and as a brief, non-technical summary of the main results and
their implications. The abstract must not include subheadings (unless
expressly permitted in the journal's Instructions to Authors), equations
or citations. As a guide the abstract should not exceed 200 words. Most
journals do not set a hard limit however authors are advised to check
the author instructions for the journal they are submitting to.

\textbf{Results:} The abstract serves both as a general introduction to
the topic and as a brief, non-technical summary of the main results and
their implications. The abstract must not include subheadings (unless
expressly permitted in the journal's Instructions to Authors), equations
or citations. As a guide the abstract should not exceed 200 words. Most
journals do not set a hard limit however authors are advised to check
the author instructions for the journal they are submitting to.

\textbf{Conclusion:} The abstract serves both as a general introduction
to the topic and as a brief, non-technical summary of the main results
and their implications. The abstract must not include subheadings
(unless expressly permitted in the journal's Instructions to Authors),
equations or citations. As a guide the abstract should not exceed 200
words. Most journals do not set a hard limit however authors are advised
to check the author instructions for the journal they are submitting
to.\}}

\keywords{key, dictionary, word}


\pacs[JEL Classification]{D8, H51}
\pacs[MSC Classification]{35A01, 65L10}

\maketitle

\newpage

\hypertarget{resumo}{%
\section{Resumo}\label{resumo}}

Este trabalho apresenta um estudo de simulação sobre métodos de
amostragem complexa para a estimação da média amostral. Os métodos
analisados são baseados em amostragem conglomerada em 1, 2 e 3 estágios.
O objetivo é comparar o desempenho desses métodos em termos de
eficiência e precisão da estimativa.

A amostragem complexa é amplamente utilizada em pesquisas em que a
população de interesse possui uma estrutura hierárquica ou está dividida
em subpopulações distintas. A amostragem conglomerada é uma técnica
comumente aplicada nesse contexto, em que a população é dividida em
conglomerados e, em seguida, uma amostra é selecionada em cada
conglomerado.

Neste estudo, são simulados diferentes cenários com base em parâmetros
de amostragem realistas. São considerados os métodos de amostragem
conglomerada em 1, 2 e 3 estágios, nos quais a seleção dos conglomerados
e das unidades amostrais é realizada de forma sequencial.

Através das simulações, são comparados os estimadores das médias
amostrais obtidos pelos diferentes métodos, levando em consideração a
variância da estimativa e a eficiência em relação ao tamanho da amostra.
Além disso, são avaliados possíveis vieses de estimadores e a precisão
das estimativas em cada estágio da amostragem conglomerada.

Os resultados das simulações fornecem insights valiosos sobre a
adequação e o desempenho dos métodos de amostragem conglomerada em
diferentes estágios. Espera-se que este estudo contribua para a
compreensão das complexidades da amostragem em pesquisas com estrutura
hierárquica e auxilie pesquisadores na escolha do método de amostragem
mais apropriado para suas necessidades.

\hypertarget{introduuxe7uxe3o}{%
\section{Introdução}\label{introduuxe7uxe3o}}

A amostragem desempenha um papel fundamental na estatística, permitindo
aos pesquisadores obterem informações sobre uma população a partir de
uma amostra representativa. Através de métodos estatísticos robustos, é
possível extrapolar conclusões precisas e confiáveis sobre a população
em geral. No entanto, a amostragem muitas vezes enfrenta desafios
práticos, como a seleção adequada das unidades amostrais e a
consideração de complexidades inerentes a certos planos de amostragem.

De forma geral, é amplamente reconhecido na teoria da amostragem que,
embora o esquema de amostragem aleatória simples (AAS) seja teoricamente
simples, na prática, é pouco utilizado devido às restrições
orçamentárias e à busca por métodos probabilísticos que forneçam
informações mais precisas. Além disso, é comum encontrar dificuldades na
obtenção de cadastros adequados para o AAS, bem como lidar com situações
de não resposta, o que requer considerar observações com pesos desiguais
\citep{skinner2005design}. A especificação inadequada na análise do
plano amostral selecionado também pode resultar em estimativas
enviesadas, destacando a importância de estudar metodologias que levem
em conta o esquema de amostragem adotado.

Este artigo tem como objetivo explorar a interseção entre a amostragem
em estatística e a simulação computacional, destacando como essa
abordagem combinada pode contribuir para aprimorar a qualidade das
inferências estatísticas. Serão apresentados conceitos fundamentais da
amostragem, incluindo diferentes métodos de seleção amostral e as
respectivas propriedades, e, em seguida, será discutido como a simulação
computacional pode ser aplicada para investigar essas técnicas em
contextos específicos.

Ao integrar a simulação computacional à amostragem estatística, os
pesquisadores podem explorar virtualmente uma ampla gama de cenários de
amostragem, considerando diferentes planos amostrais, tamanhos de
amostra e distribuições populacionais. Além disso, a simulação permite a
avaliação de métricas de desempenho, como viés e erro padrão, fornecendo
insights valiosos sobre a precisão e a eficiência dos métodos de
amostragem em diferentes contextos.

\hypertarget{metodologia}{%
\section{Metodologia}\label{metodologia}}

O objetivo deste trabalho é comparar diferentes planos de amostragem em
estágios complexos, como a amostragem estratificada e a amostragem
conglomerada. Para realizar essa comparação, foi conduzido um estudo de
simulação. O estudo tem como propósito investigar e avaliar o desempenho
desses diferentes planos amostrais em termos de eficiência, precisão e
viés. Através da simulação, é possível criar cenários controlados que
permitem analisar o impacto de cada plano amostral em diferentes
características da população. Com base nos resultados obtidos na
simulação, será possível identificar quais planos de amostragem são mais
adequados para determinados contextos e auxiliar na tomada de decisões
estatísticas mais embasadas.

Para isso, foi utilizado o conjunto de dados: \textbf{Alunos.txt}, que
se trata de dados sobre notas de alunos na prova de portugues. Os dados
são populacionnais, ou seja, é um cadastro completo dos alunos da rede
básica de **** lugar. Assim, os diferentes métodos de amostragem
complexa foram utilizados em cima desse conjunto de dados.

O conjunto de dados possui 6 variáveis:

\begin{itemize}
\tightlist
\item
  Aluno

  \begin{itemize}
  \tightlist
  \item
    Se trata de um ID individual para cada observação no cadastro
  \end{itemize}
\item
  Rede

  \begin{itemize}
  \tightlist
  \item
    Se trata de um ID para cada rede de ensino no cadastro, cada rede
    pode possuir mais de uma escola
  \end{itemize}
\item
  Escola

  \begin{itemize}
  \tightlist
  \item
    Se trata de um ID para cada escola no cadastro
  \end{itemize}
\item
  Turma

  \begin{itemize}
  \tightlist
  \item
    Se trata de um ID para cada turma no cadastro
  \end{itemize}
\item
  Port

  \begin{itemize}
  \tightlist
  \item
    Se trata da nota no teste de portugues de cada aluno, é tambem a
    variável de interesse desse trabalho
  \end{itemize}
\end{itemize}

Os metodos utilizados foram:

\begin{itemize}
\item
  Amostragem Estratificada

  \begin{itemize}
  \tightlist
  \item
    Foram testados estratificação por Rede e estratificação por Escola
  \end{itemize}
\item
  Amostragem Conglomerada

  \begin{itemize}
  \item
    1 estágio por Escolas
  \item
    1 estágio por Turmas
  \item
    2 estágios: UPA-Escolas, USA-Turmas
  \item
    3 estágios: UPA-Escolas, USA-Turmas, UTA-Alunos
  \end{itemize}
\item
  Amostragem Conglomerada com PPT Poisson

  \begin{itemize}
  \tightlist
  \item
    1 estágio por Escolas, tamanho via número de turmas
  \item
    1 estágio por Escolas, tamanho via número de alunos
  \end{itemize}
\end{itemize}

\hypertarget{estudo-de-simulauxe7uxe3o}{%
\section{Estudo de Simulação}\label{estudo-de-simulauxe7uxe3o}}

Considerou-se como variavel de interesse a média da variavel Port com
transformação logaritmo natural, ou seja, a variavel estimada via
diferentes metodos de amostragem complexa foi: \(ln({Port})\)

Para a cada plano amostral, foram replicadas 1000 vezes amostras de
tamanho 500 e 1000 vezes amostras de tamanho 750, para cada \emph{pool}
foram calculadas as estimativas pontuais, o erro padrão e o intervalo de
confiança de \(95\%\)

Para avaliar o desempenho de cada plano amostral. foram consideradas
metricas como \textbf{Víes, Erro-padrão e Erro Quadrático Médio}.

O verdadeiro valor da variavel estimada é de:

\[
\frac{\sum_{i=1}^n{ln(Port_i)}}{n} = 6.218181
\]

O conhecimento de tal valor é importantíssimo para o calculo do viés e
consequente a decisão sobre o plano maostral mais adequado para o
problema.

\hypertarget{amostragem-estratificada}{%
\subsection{Amostragem Estratificada}\label{amostragem-estratificada}}

\hypertarget{estratificada-por-rede}{%
\subsubsection{Estratificada por Rede}\label{estratificada-por-rede}}

Temos as seguintes estatísticas após 1000 replicações:

\begin{verbatim}
##   Estimativas  ErroPadrão IntContReal          Vies          EQM
## 1    6.218338 0.010457632       0.966  1.567962e-04 1.093866e-04
## 2    6.218294 0.009088516       0.951  1.121201e-04 8.261370e-05
## 3    6.218179 0.009084075       0.964 -2.021792e-06 8.252042e-05
\end{verbatim}

\begin{verbatim}
##   Estimativas  ErroPadrão IntContReal          Vies          EQM
## 1    6.217622 0.008560626       0.968 -5.596679e-04 7.359755e-05
## 2    6.218224 0.007427772       0.960  4.267254e-05 5.517361e-05
## 3    6.217922 0.007432379       0.966 -2.591951e-04 5.530744e-05
\end{verbatim}

\hypertarget{estratificada-por-escola}{%
\subsubsection{Estratificada por
Escola}\label{estratificada-por-escola}}

\begin{verbatim}
##   Estimativas  ErroPadrão IntContReal          Vies          EQM
## 1    6.218014 0.009482171       0.981 -0.0001673409 8.993958e-05
## 2    6.217734 0.008404116       0.983 -0.0004475129 7.082943e-05
## 3    6.217978 0.008674326       0.986 -0.0002035119 7.528534e-05
\end{verbatim}

\begin{verbatim}
##   Estimativas  ErroPadrão IntContReal          Vies          EQM
## 1    6.218394 0.008176467       0.982  2.124853e-04 6.689976e-05
## 2    6.218155 0.007064401       0.991 -2.680149e-05 4.990648e-05
## 3    6.218010 0.007205262       0.984 -1.718501e-04 5.194533e-05
\end{verbatim}

\hypertarget{amostragem-conglomerada}{%
\subsection{Amostragem Conglomerada}\label{amostragem-conglomerada}}

\hypertarget{conclusuxe3o}{%
\section{Conclusão}\label{conclusuxe3o}}

\renewcommand\refname{Referências}
\bibliography{bibliography.bib}


\end{document}
